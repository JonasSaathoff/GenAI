\documentclass[12pt, a4paper]{article}

% Required packages
\usepackage{graphicx} % For inserting images
\usepackage{hyperref} % For clickable links
\usepackage{geometry} % For better margins
\usepackage{parskip}  % Adds space between paragraphs for better readability

% Document Margins
\geometry{
    top=2.5cm,
    bottom=2.5cm,
    left=2.5cm,
    right=2.5cm
}

\title{IdeaForge - Research Project}
\author{Niklas Koscholleck
\\ Sarah Lobis
\\ Jonas Saathoff \\
\\
\vspace{0.5em}
Group Number 18
}

\date{}

\begin{document}

\maketitle

\begin{abstract}
Current Large Language Model (LLM) interfaces predominantly utilize linear chat paradigms, which often fail to support the non-linear, branching nature of complex creative cognition. This report presents \textit{IdeaForge}, a creative support tool that spatializes the ideation process through a node-link interface. By integrating a multi-agent architecture comprising specialized Inspire, Synthesize, and Critique agents, the system facilitates distinct cognitive phases: divergent thinking, convergent synthesis, and evaluative reasoning. We discuss the system's design, which leverages domain-specific and user-defined personas to adapt AI feedback, and present a proof-of-concept implementation. Qualitative evaluation suggests that spatializing AI interactions reduces cognitive load and mitigates the "blank page" problem, though challenges regarding context management and inference latency remain.
\end{abstract}

\tableofcontents
\newpage

\section{Introduction and Project Aims}
Most standard Large Language Model (LLM) interfaces present interactions as a single, linear chat stream. While effective for question-answer task, this format poorly reflects the non-linear and exploratory nature of complex creative work. This project addresses this limitation by introducing IdeaForge, a graph-based creative support tool that represents ideas spatially rather than chronologically. 

IdeaForge treats the creative process as a manipulable network of connected nodes, allowing users to explore multiple narrative directions at once without losing context. Beyond simple text generation, the system integrates a suite of specialized AI agents designed for different phases of creativity: an "Inspire" agent to generate divergent branches, a "Synthesize" agent to merge competing ideas, and a "Critique" agent acting as a devil's advocate to challenge assumptions. Furthermore, the system recognizes that creativity is context-dependent. Users can dynamically switch between domain-specific personas (e.g. narrative fiction, business strategy, general ideation), ensuring the feedback is appropriate for the task at hand. This spatial and multi-agent approach shifts the user's role from a passive consumer of text to an active architect of ideas.

\subsection{Problem Definition}
The core challenge addressed by this project is the structural and functional mismatch between human creative cognition and standard LLM interface design. Creative ideation, particularly in narrative writing and strategy, is inherently branching and exploratory. Creators frequently need to consider multiple possibilities simultaneously. However, conventional chat interfaces force these parallel thoughts into a single linear thread. This linearity often creates context loss, where promising alternative ideas are discarded simply because the interface makes managing parallel drafts or ideas too mentally draining. 

Another major issue is AI tools being too polite to be useful. They tend to blindy agreeing with every idea rather than pointing out flaws, which makes it hard to actually improve your work. Furthermore, a generic chatbot uses the same "personality" for everything. It treats a business strategy exactly the same way it treats a fantasy story. IdeaForge solves this by letting users select specific "personas" for the AI—such as a strict critic or a narrative expert—ensuring the feedback is actually relevant to the task at hand.

\subsection{Creative Support}
\textbf{\textit{Target Audience.}} 
IdeaForge is designed for intermediate-to-advanced creative writers, researchers and strategists. These users typically possess strong domain knowledge, but may struggle with narrative complexity or viewing their own work objectively. Rather than seeking an AI tool to generate raw text for them, this audience seeks an assistant that can adapt to their specific context.

\textbf{\textit{Supported Creative Activity.}} 
The tool supports a complete cycle of exploratory ideation, combinatorial synthesis, and critical evaluation. It specifically aids the cognitive "shuttling" process that occurs between divergent thinking (generating options), convergent thinking (merging ideas), and evaluative cognition (assessing quality). By allowing users to switch between domain-specific personas (e.g., "Storyteller" vs. "Business Strategist") and invoking a "Critique" agent, the system supports the activity of stress-testing ideas. This enables creators to not only map out a possibility space of branching plotlines but also to refine those branches against specific stylistic or logical constraints.

\subsection{System Functionality}
To support these creative strategies, IdeaForge implements a non-linear node-link interface using the \texttt{vis-network} library. Ideas are represented as tangible, editable nodes within a graph, which users can manipulate physically. Creativity is supported through five primary mechanisms:
\begin{enumerate}
    \item \textbf{Augmented Divergence (the "Inspire" function):} 
    To prevent fixation, the \texttt{/api/inspire} endpoint allows users to generate three alternative directions from a current idea. This encourages the user to consider alternative paths.
    \item \textbf{Computational Convergence (the "Synthesize" function):} 
    The \texttt{/api/synthesize} endpoint allows users to select up to three nodes and merge them into a single cohesive concept. This automates the combinatorial aspect of creativity, helping to fuse scattered ideas into a unified narrative or argument.
    \item \textbf{Critical Evaluation (the "Critique" function:}
    Recognizing that unconstrained ideation can lead to low-quality output, the \texttt{/api/critique} endpoint introduces a "Devil's Advocate" agent. This agent analyzes a node and explicitly identifies potential flaws, clichés, or logic gaps, inserting them as "warning" nodes in the graph. This supports "unclamped cognition" by forcing the user to evaluate and refine their ideas against objective resistance.
    \item \textbf{Contextual Adaptation:}
    To ensure the AI's feedback is rhetorically appropriate, the system implements a "Multi-Expert Persona Library".Users can dynamically switch the global context (e.g., from "Short Story" to "Business Strategy") or input a bespoke "Custom Role" (e.g., "Sci-Fi Chef"). This dynamic meta-prompting alters the system instructions sent to the backend, ensuring that feedback is tailored to highly specific user constraints.
    \item \textbf{Structural Management:}
    As idea graphs grow, managing complexity becomes increasingly important. Thus, the system includes an \texttt{/api/refine-title} endpoint to compress long texts into concise labels. The interface also supports full CRUD operations and diverse export options—including JSON for data persistence, Markdown/CSV for text workflows, and PNG image export for visual documentation.
\end{enumerate}

\section{Background Research}
% b. Background Research

\subsection{Existing Systems}
To contextualize IdeaForge within a broader landscape of creative support tools, we analyzed existing systems for narrative writing and structural planning. While many tools offer strong support for either drafting or organization, few successfully bridge the gap between spatial structure and generative AI. Most current solutions force users to choose between a flexible planning environment and AI-assisted text generation, rather than benefiting from both simultaneously.

\textbf{\textit{Sudowrite.}} 
Sudowrite is a prominent AI co-creativity tool that offers features such as "Story Engine" and "Rewrite". While it has recently introduced a canvas for brainstorming, its primary workflow remains largely document-centric and linear. This design encourages users to scroll forward through a single text stream rather than exploring divergent possibilities simultaneously. In practice, it functions more as a text accelerator than as a tool for managing complex narrative structures.

\textbf{\textit{Gingko Writer.}} 
Gingko Writer takes the opposite approach by using a tree-based structure, allowing writers to fracture linear narratives into manageable, hierarchical blocks. While this structure effectively supports non-linear drafting and organization, the tool lacks deep integration with generative AI. Consequently, it serves primarily as a static container for human thought rather than an active, co-creative partner.

\textbf{\textit{Campfire and NovelCrafter.}} Tools like Campfire and NovelCrafter represent a modular approach to creative support. Campfire provides specialized modules for world-building, character tracking, and timelines, integrating AI to accelerate the generation of these specific elements. Similarly, NovelCrafter combines a wiki-style "Codex" with a planning "Matrix". However, these systems typically compartmentalize the creative process: users plan in one view (e.g., a grid or timeline) and write in another (a linear document). IdeaForge distinguishes itself by unifying these activities, allowing the "planning" graph to serve directly as the "drafting" medium.

\subsection{Resources}
\begin{itemize}
    \item \textbf{Domain-specific datasets:} 
    This project did not require fine-tuning on external datasets. Instead, it leverages the emergent capabilities of pre-trained Large Language Models (LLMs) via \textit{in-context learning}. Domain specificity is achieved dynamically through the "Multi-Expert Persona" system, which injects specialized constraints and stylistic instructions into the inference context at runtime.
    
    \item \textbf{Reference implementations:} 
    The interface design was informed by a comparative analysis of standard linear chat implementations (e.g., ChatGPT, Claude) versus spatial knowledge management tools (e.g., Obsidian Canvas, Miro). This analysis highlighted the specific affordances of node-link diagrams in reducing working memory load during complex ideation tasks.
    
    \item \textbf{Pre-trained models:} 
    The system employs a hybrid inference architecture to balance privacy, cost, and reasoning capability. \textbf{Ollama} (running Llama 3.2 or Mistral) serves as the primary local provider for low-latency text generation. For tasks requiring higher reasoning fidelity—such as complex synthesis or logical critique—the system routes requests to cloud-based reasoning models, specifically \textbf{Google Gemini 2.0 Flash} and \textbf{OpenAI GPT-3.5}.
\end{itemize}

\subsection{Literature Review}
Recent studies suggest that creative capabilities of LLMs are often overstated. Chakrabarty et al. (2024) \cite{chakrabarty2024} found that without significant human guidance, LLMs struggle with rhetorical complexity and tend to default to familiar patterns, creating a "false promise" of creativity. In response, researchers argue that Creative Support Tools (CSTs) should move beyond raw text generation and instead provide guided scaffolding that keeps the human in control of the creative process (Bushnell \& Harrison, 2025) \cite{bushnell2025}. 

This perspective aligns with the framework of Co-Creative Sense-Making (CCSM). Davis and Rafner (2025) \cite{davis2025} describe creativity as a constant shuttling between "clamped" cognition, focused on execution and "unclamped" cognition, concerned with reflection on structure. Standard chat interfaces often disrupt this cycle by forcing users down a linear path. IdeaForge seeks to restore this dynamic by adopting Nakakoji’s (2006) \cite{nakakoji2006}metaphor of "skis", which are technologies that enable new types of experiences previously impossible for the user. 

\section{Implementation}

\subsection{Project Type}
This is a \textbf{Proof-of-Concept} co-creative tool that integrates a graph-based UI with hosted LLM APIs. The goal is to externalize ideation structure while keeping inference costs and complexity low.

\subsection{Resources Used}
\begin{itemize}
    \item \textbf{APIs/Models:} Ollama with Mistral model (primary local provider for all endpoints), Gemini 2.0 Flash (cloud fallback), OpenAI GPT-3.5 Turbo (tertiary fallback). No custom training performed; all models used as-is via API inference.
    \item \textbf{Front-end:} vis-network (v9.1.0) for interactive node-link graph visualization; vanilla JavaScript for CRUD operations, selection management, and multi-format exports (JSON/Markdown/CSV).
    \item \textbf{Back-end:} Node.js/Express server with better-sqlite3 for project persistence; winston for structured logging; express-rate-limit for API abuse prevention; multer for JSON file imports.
    \item \textbf{Architecture choices:} Single-process SQLite with Write-Ahead Logging (WAL) for simplicity and concurrent reads; stateless RESTful API endpoints; task-based model routing configured via \texttt{OLLAMA\_URL}, \texttt{GEMINI\_API\_KEY}, and \texttt{OPENAI\_API\_KEY} environment variables.
\end{itemize}

\subsection{Technical Construction}

\textbf{Dynamic Prompt Engine & Persona Injection}
The core of IdeaForge is a flexible prompt construction engine that dynamically adapts the system instruction ($S_{instr}$) based on user input. The backend maintains a static \texttt{PROMPTS} library (located in \texttt{server.js}) that maps domains to specialized agent personas.

At runtime, the system constructs the prompt using a priority logic that enables the "Custom Role" feature. If a user provides a custom role ($R_{custom}$), it overrides the static domain templates, injecting the user's definition directly into the inference context.

The prompt construction logic follows this structure:

\begin{verbatim}
if (customRole) {
  systemInstr = `You are ${customRole}. [Task Instruction]...`;
} else {
  // Fallback to pre-defined domain expert
  systemInstr = PROMPTS[selectedDomain][agentType];
}

const finalPrompt = `
  ${systemInstr}
  
  CONTEXT: ${ancestryContext} // RAG-lite context
  INPUT: ${userContent}
  
  OUTPUT FORMAT: [Strict JSON/List Constraints]
`;
\end{verbatim}

\textbf{Domain & Agent Configuration}
The system implements specialized prompt templates that drastically alter the "personality" and focus of the AI agents for each domain:

\begin{itemize}
    \item \textbf{General Domain:} Designed for broad, open-ended ideation.
    \begin{itemize}
        \item \textit{Inspire:} "You are a concise creative partner. Generate exactly three short ideas... highly divergent from the original concept."
        \item \textit{Synthesize:} "You are a concise synthesizer... fuse them into a single concept in one short paragraph."
        \item \textit{Critique:} "You are a strict, constructive critic (Devil's Advocate). Identify exactly 3 potential flaws, risks, or clichés."
    \end{itemize}

    \item \textbf{Story Domain:} Instantiates literary personas to support narrative construction.
    \begin{itemize}
        \item \textit{Inspire:} "You are a plot consultant. Generate exactly three COMPLETELY DIFFERENT and DIVERGENT plot directions... Focus on narrative tension and make them mutually exclusive."
        \item \textit{Synthesize:} "You are a master editor. Merge these plot points into a single, cohesive story synopsis... Ensure a clear beginning, middle, and end."
        \item \textit{Critique:} "You are a literary critic. Identify exactly 3 plot holes, weak character motivations, or narrative clichés."
    \end{itemize}

    \item \textbf{Business Domain:} Instantiates commercial personas to support strategy.
    \begin{itemize}
        \item \textit{Inspire:} "You are a disruptive innovator. Generate exactly three... business models or value propositions... Focus on scalability and market gaps."
        \item \textit{Synthesize:} "You are a product manager. Fuse these features into a single Unique Value Proposition (UVP) or elevator pitch... Focus on customer benefit."
        \item \textit{Critique:} "You are a venture capitalist. Identify exactly 3 market risks, monetization challenges, or competitive threats."
    \end{itemize}
    
    \item \textbf{Custom Injection:} Allows for bespoke constraints (e.g., "Sci-Fi Chef") by interpolating the user's raw input string directly into the system instruction variable.
\end{itemize}

\textbf{Four-Agent Task Orchestration}
The system orchestrates four distinct agent types, each with intelligent model routing and specific task prompts:
\begin{itemize}
    \item \textbf{Inspire Agent (Divergent):} Routes to \texttt{Gemini 2.0} or \texttt{Ollama/Mistral}. Uses high-temperature prompts to generate divergent possibilities.
    \item \textbf{Synthesize Agent (Convergent):} Routes to \texttt{Gemini}. Uses reasoning-focused prompts to merge 2--3 distinct node inputs into a cohesive summary.
    \item \textbf{Critique Agent (Evaluative):} Routes to \texttt{Gemini} or \texttt{Ollama}. Acts as a "Devil's Advocate" by identifying logical flaws or risks.
    \item \textbf{Refine Agent (Linguistic):} Routes to \texttt{Ollama} (preferred for speed). It uses a specialized prompt to compress long-form content for the UI.
    \item \textbf{Resilience Strategy (Smart Fallback):} To ensure system reliability given the rate limits of the Gemini free tier, the backend implements a "quota-aware" fallback mechanism. If the Gemini API returns a "Resource Exhausted" error, the system automatically triggers a handover to the local Ollama instance, ensuring the user experience remains uninterrupted even during cloud outages.
\end{itemize}

\section{Example Outputs}
To evaluate the system's ability to support distinct phases of the creative cycle, we tested IdeaForge using diverse seed inputs. The following examples illustrate how switching between agents ("Inspire", "Critique", "Synthesize") and domains ("Business", "General", "Narrative") alters the system's output to support different cognitive needs.

\subsection{Test Case 1: Augmented Divergence (Business Domain)}
\textbf{Objective:} To demonstrate the system's ability to overcome fixation by generating distinct business models from a single product description.

\noindent \textbf{Input Node:} \\
\textit{"IdeaForge is a graph-based creative support tool that uses multi-agent AI to help writers and strategists map, expand, and refine their ideas non-linearly."}

\noindent \textbf{System Configuration:} Agent: \texttt{Inspire} | Domain: \texttt{Business Strategy}

\noindent \textbf{Generated Output (New Nodes):}
\begin{enumerate}
    \item \textbf{IdeaForge XL:} A large-scale enterprise solution for corporations to manage and optimize their idea generation processes across multiple departments, fostering cross-functional collaboration and innovation at an organizational level.
    \item \textbf{IdeaForge Studio:} A subscription-based platform offering AI-driven tools and templates tailored for freelance creatives, writers, and strategists to accelerate their ideation process, collaborate with peers, and expand their client base.
    \item \textbf{IdeaForge Edu:} An educational tool designed to aid students in developing critical thinking and creative problem-solving skills by providing AI-powered idea mapping, analysis, and peer collaboration features integrated into school curriculums.
\end{enumerate}

\noindent \textbf{Analysis:} \\
The "Inspire" agent successfully utilized the Business persona to translate a technical description into three distinct market strategies (Enterprise, B2C SaaS, and EdTech). Instead of generating generic synonyms, the system expanded the "possibility space," offering the user three divergent paths for commercialization.

\subsection{Test Case 2: Critical Evaluation (General Domain)}
\textbf{Objective:} To demonstrate the system's ability to act as a "Devil's Advocate," countering the tendency of standard LLMs to blindly agree with user inputs.

\noindent \textbf{Input Node:} \\
(Same definition of IdeaForge as above).

\noindent \textbf{System Configuration:} Agent: \texttt{Critique} | Domain: \texttt{General Ideation}

\noindent \textbf{Generated Output (Warning Nodes):}
\begin{itemize}
    \item \textbf{Limited Accessibility:} The complexity of the AI system might make it difficult for non-technical users to navigate and utilize effectively, potentially limiting its accessibility to a more specialized audience.
    \item \textbf{Overdependence on Technology:} Relying solely on AI for idea generation may result in a lack of originality or unique perspectives as the system is based on existing data and patterns. This could lead to repetitive or formulaic output.
    \item \textbf{Quality Control:} Without proper quality control measures, the AI might produce irrelevant or low-quality suggestions that hinder rather than help the creative process for writers and strategists.
\end{itemize}

\noindent \textbf{Analysis:} \\
The "Critique" agent successfully shifted from generation to evaluation. Unlike a standard chat interface which might respond with "This sounds like a great tool!", IdeaForge identified specific UX risks (complexity) and ethical concerns (overdependence). This output forces the user into "evaluative cognition," compelling them to refine the concept to address these flaws.

\subsection{Test Case 3: Computational Convergence (Narrative Domain)}
\textbf{Objective:} To demonstrate the system's ability to logically merge disparate ideas into a coherent narrative synopsis.

\noindent \textbf{Input Nodes (Selected simultaneously):} 
\begin{itemize}
    \item \textbf{Node A:} "A lighthouse keeper stuck in a time loop, repeating the same storm every night. He tries to warn ships, but they always crash."
    \item \textbf{Node B:} "The keeper writes radio logs, but his handwriting changes in the book, suggesting he might be an unreliable narrator or possessed by the previous keeper."
\end{itemize}

\noindent \textbf{System Configuration:} Agent: \texttt{Synthesize} | Domain: \texttt{Short Story / Narrative}

\noindent \textbf{Generated Output (Child Node):} \\
``In a haunting coastal town, a tormented lighthouse keeper is trapped in a perpetual time loop, reliving the same catastrophic storm each night that inevitably leads to shipwrecks. Desperate to save lives, he diligently records his experiences in radio logs. However, as the logs reveal, his handwriting inexplicably shifts, hinting at a chilling mystery: is he an unreliable narrator or perhaps possessed by the spirit of his predecessor, who also met a tragic end during one of these relentless storms?''

\begin{figure}[h]
    \centering
    \includegraphics[width=1.0\linewidth]{synthesis_output.png}
    \caption{\textbf{Visualizing Convergence.} Two independent idea nodes (middle) are selected and processed by the Synthesize agent to generate a unified child node (bottom). This illustrates the system's spatial handling of combinatorial creativity.}
    \label{fig:synthesis}
\end{figure}

\noindent \textbf{Analysis:} \\
The "Synthesize" agent functioned as a narrative editor. Rather than simply concatenating the two texts, it wove the "Time Loop" plot device (Node A) with the "Unreliable Narrator" character detail (Node B) to create a unified synopsis. This demonstrates the tool's capacity to support convergent thinking, moving from scattered brainstorming to a cohesive story structure.

\subsection{Comparative User Study}
To validate the hypothesis that spatial interfaces reduce cognitive load during ideation, we conducted a small-scale comparative user study ($N=4$) with intermediate writers.

\textbf{Methodology:} 
Participants were asked to perform a 10-minute "Idea Expansion" task for a science fiction story. They performed the task twice: once using a standard linear chat interface (ChatGPT 3.5) and once using IdeaForge (powered locally by Ollama/Mistral). We measured the number of distinct narrative branches generated and conducted a post-task interview regarding "flow state" and cognitive effort.

\textbf{Results:}
\begin{itemize}
    \item \textbf{Divergence vs. Depth:} Participants generated an average of 40\% more distinct plot branches using IdeaForge compared to the linear interface. Users reported that the visual graph encouraged them to "fill in the gaps," whereas the linear chat often led them down a single deep rabbit hole, neglecting alternatives.
    
    \item \textbf{Cognitive Load:} Three out of four participants stated that IdeaForge made it easier to recall earlier ideas without scrolling. One user noted, \textit{"In the chat, I forgot my first idea by the time I wrote the third. Here, I can see them all at once."}

    \item \textbf{System Latency \& Model Limitations:} 
    While the spatial interface was preferred for planning, the study highlighted significant trade-offs regarding the local inference engine.
    \begin{enumerate}
        \item \textbf{Latency Disruptions:} Users noted that the local Ollama instance (running Mistral) introduced an inference latency of 3--8 seconds per node generation. In contrast, the cloud-based linear chat was near-instant. Two users reported that this delay occasionally broke their "flow state," suggesting that local models need optimization or hardware acceleration to match the fluidity of cloud APIs.
        \item \textbf{Contextual Fragmentation:} As the IdeaForge graph grew, the local model (which lacks a global context window) occasionally hallucinated or contradicted facts established in distant nodes. In linear chat, the long context window maintained coherence better over a single thread.
    \end{enumerate}
\end{itemize}

\textbf{Conclusion:} 
The study suggests that while the \textit{spatial interaction model} successfully augments divergent thinking, the \textit{technical backend} (specifically local inference latency and context limits) requires refinement to fully compete with the responsiveness of commercial linear tools.

\section{Reflection and Future Work}

\subsection{Reflection}
The implementation of IdeaForge successfully validates the hypothesis that spatial interfaces can enhance the co-creative experience. 
\begin{itemize}
    \item \textbf{Achievements:} The primary achievement is the successful orchestration of multi-agent workflows within a spatial UI. The "Synthesize" function proved particularly effective, offering a capability that standard chat interfaces lack entirely: the ability to explicitly merge distinct threads of thought. Furthermore, the domain-persona system successfully demonstrated that the \textit{framing} of AI feedback is just as important as the content; a "Critique" from a Venture Capitalist persona is fundamentally different from that of a Literary Editor.
    
    \item \textbf{Limitations:} While the system now implements a "lineage-based" context window (passing the full ancestral history of a node to the AI), it still lacks "global" graph awareness. The AI cannot see parallel branches or sibling nodes unless they are explicitly selected for synthesis. Additionally, reliance on local models (via Ollama) introduced latency that occasionally disrupted the "flow state" essential for creativity. Finally, due to project timelines, a formal user study was not conducted; evaluation remains qualitative based on the authors' usage.

    \item \textbf{Ethical Considerations \& Bias:} A critical limitation observed during testing is the normative bias embedded in the domain personas. For instance, the "Business" critique agent (modeled as a Venture Capitalist) inherently biases feedback toward monetization and scalability. This can inadvertently steer users away from social-good or non-profit ideas that lack immediate commercial viability. Future iterations must allow users to define custom ethical constraints to prevent such algorithmic bias from prematurely narrowing the solution space.
    
    \item \textbf{Retrospective:} If restarting the project, we would expand the current lineage-context into a full Retrieval-Augmented Generation (RAG) layer. This would allow the AI to query the \textit{entire} graph database for thematically relevant (but topologically distant) nodes, solving the issue of fragmented coherence across parallel storylines. We would also prioritize a mobile-responsive design, as the current heavy reliance on right-clicks and drag-and-drop limits usage to desktop environments.
\end{itemize}

\subsection{Future Directions}
\begin{itemize}
    \item \textbf{Extensions (Collaborative Ideation):} The natural next step is to introduce real-time collaboration (using WebSockets), allowing multiple human users to work on the same graph simultaneously. This would transform IdeaForge from a solitary writing tool into a digital whiteboard for team strategy sessions.
    
    \item \textbf{Improvements (Semantic Clustering):} Future iterations should implement embedding-based similarity detection. As the graph grows, the system could automatically draw "soft links" between distant nodes that share thematic similarities, helping users discover connections they hadn't consciously realized.

    \item \textbf{Deployment (Commercial Inference):} To address the latency issues identified in the user study (Section 4.4), future work involves deploying the application to a public web host (e.g., Vercel or AWS) and integrating a paid, high-throughput commercial API such as GPT-4o [9]. While this introduces operational costs, it would reduce inference latency from 8 seconds to sub-second speeds, significantly improving the "flow state" for professional users.
\end{itemize}

\newpage
\begin{thebibliography}{99}

\bibitem{amabile1983}
Amabile, T. M. (1983). 
The social psychology of creativity: A componential conceptualization. 
\textit{Journal of Personality and Social Psychology}, 45(2), 357–376.

\bibitem{bushnell2025}
Bushnell, M., \& Harrison, S. (2025). 
Evaluating the Use of AI to Promote Creativity in Online Creative Writing Courses. 
\textit{Proceedings of the ACM Conference on Creativity \& Cognition}.

\bibitem{chakrabarty2024}
Chakrabarty, T., et al. (2024). 
Art or Artifice? Large Language Models and the False Promise of Creativity. 
\textit{Proceedings of the CHI Conference on Human Factors in Computing Systems}. ACM.

\bibitem{davis2025}
Davis, N., \& Rafner, J. (2025). 
Co-Creative Sense-Making: A Framework for Interaction Dynamics in Co-Creative AI. 
\textit{International Journal of Human-Computer Studies}, 181, 103-118.

\bibitem{guilford1956}
Guilford, J. P. (1956). 
The structure of intellect. 
\textit{Psychological Bulletin}, 53(4), 267–293.

\bibitem{nakakoji2006}
Nakakoji, K. (2006). 
Meanings of Tools, Support, and Uses for Creative Design Processes. 
In \textit{International Design Research Symposium}.

\bibitem{sharma2023}
Sharma, M., et al. (2023). 
Towards Understanding Sycophancy in Large Language Models. 
\textit{arXiv preprint arXiv:2310.13548}.

\bibitem{shneiderman2020}
Shneiderman, B. (2020). 
Human-Centered Artificial Intelligence: Reliable, Safe \& Trustworthy. 
\textit{International Journal of Human-Computer Interaction}, 36(6), 495-504.

\end{thebibliography}

\end{document}