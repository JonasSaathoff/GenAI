\documentclass[12pt, a4paper]{article}

% Required packages
\usepackage{graphicx} % For inserting images
\usepackage{hyperref} % For clickable links
\usepackage{geometry} % For better margins
\usepackage{parskip}  % Adds space between paragraphs for better readability

% Document Margins
\geometry{
    top=2.5cm,
    bottom=2.5cm,
    left=2.5cm,
    right=2.5cm
}

\title{Generative AI Project Report}
\author{Your Name}
\date{January 2026}

\begin{document}

\maketitle

\tableofcontents
\newpage

\section{Introduction and Project Aims}
Most standard Large Language Model (LLM) interfaces present interactions as a single, linear chat stream. While effective for question-answer task, this format poorly reflects the non-linear and exploratory nature of complex creative work. This project addresses this limitation by introducing IdeaForge, a graph-based creative support tool that represents ideas spatially rather than chronologically. 

IdeaForge treats the creative process as a manipulable network of connected nodes, allowing users to explore multiple narrative directions at once without losing context. Beyond simple text generation, the system integrates a suite of specialized AI agents designed for different phases of creativity: an "Inspire" agent to generate divergent branches, a "Synthesize" agent to merge competing ideas, and a "Critique" agent acting as a devil's advocate to challenge assumptions. Furthermore, the system recognizes that creativity is context-dependent. Users can dynamically switch between domain-specific personas (e.g. narrative fiction, business strategy, general ideation), ensuring the feedback is appropriate for the task at hand. This spatial and multi-agent approach shifts the user's role from a passive consumer of text to an active architect of ideas.

\subsection{Problem Definition}
The core challenge addressed by this project is the structural and functional mismatch between human creative cognition and standard LLM interface design. Creative ideation, particularly in narrative writing and strategy, is inherently branching and exploratory. Creators frequently need to consider multiple possibilities simultaneously. However, conventional chat interfaces force these parallel thoughts into a single linear thread. This linearity often creates context loss, where promising alternative ideas are discarded simply because the interface makes managing parallel drafts or ideas too mentally draining. 

Another major issue is AI tools being too polite to be useful. They tend to blindy agreeing with every idea rather than pointing out flaws, which makes it hard to actually improve your work. Furthermore, a generic chatbot uses the same "personality" for everything. It treats a business strategy exactly the same way it treats a fantasy story. IdeaForge solves this by letting users select specific "personas" for the AI—such as a strict critic or a narrative expert—ensuring the feedback is actually relevant to the task at hand.

\subsection{Creative Support}
\textbf{\textit{Target Audience.}} 
IdeaForge is designed for intermediate-to-advanced creative writers, researchers and strategists. These users typically possess strong domain knowledge, but may struggle with narrative complexity or viewing their own work objectively. Rather than seeking an AI tool to generate raw text for them, this audience seeks an assistant that can adapt to their specific context.

\textbf{\textit{Supported Creative Activity.}} 
The tool supports a complete cycle of exploratory ideation, combinatorial synthesis, and critical evaluation. It specifically aids the cognitive "shuttling" process that occurs between divergent thinking (generating options), convergent thinking (merging ideas), and evaluative cognition (assessing quality). By allowing users to switch between domain-specific personas (e.g., "Storyteller" vs. "Business Strategist") and invoking a "Critique" agent, the system supports the activity of stress-testing ideas. This enables creators to not only map out a possibility space of branching plotlines but also to refine those branches against specific stylistic or logical constraints.

\subsection{System Functionality}
To support these creative strategies, IdeaForge implements a non-linear node-link interface using the \texttt{vis-network} library. Ideas are represented as tangible, editable nodes within a graph, which users can manipulate physically. Creativity is supported through five primary mechanisms:
\begin{enumerate}
    \item \textbf{Augmented Divergence (the "Inspire" function):} 
    To prevent fixation, the \texttt{/api/inspire} endpoint allows users to generate three alternative directions from a current idea. This encourages the user to consider alternative paths.
    \item \textbf{Computational Convergence (the "Synthesize" function):} 
    The \texttt{/api/synthesize} endpoint allows users to select up to three nodes and merge them into a single cohesive concept. This automates the combinatorial aspect of creativity, helping to fuse scattered ideas into a unified narrative or argument.
    \item \textbf{Critical Evaluation (the "Critique" function:}
    Recognizing that unconstrained ideation can lead to low-quality output, the \texttt{/api/critique} endpoint introduces a "Devil's Advocate" agent. This agent analyzes a node and explicitly identifies potential flaws, clichés, or logic gaps, inserting them as "warning" nodes in the graph. This supports "unclamped cognition" by forcing the user to evaluate and refine their ideas against objective resistance.
    \item \textbf{Contextual Adaptation:}
    To ensure the AI's feedback is rhetorically appropriate, the system implements a "Multi-Expert Persona Library". Users can dynamically switch the global context (e.g., from "Short Story" to "Business Strategy" or "General Ideation"). This alters the system instructions sent to the backend, ensuring that an "Inspire" call for a story generates plot twists, while the same call for business generates value propositions.
    \item \textbf{Structural Management:}
    As idea graphs grow, managing complexity becomes increasingly important. Thus, the system includes an \texttt{/api/refine-title} endpoint to compress long texts into concise labels. The interface also supports full CRUD operations and diverse export options—including JSON for data persistence, Markdown/CSV for text workflows, and PNG image export for visual documentation.
\end{enumerate}

\section{Background Research}
% b. Background Research

\subsection{Existing Systems}
To contextualize IdeaForge within a broader landscape of creative support tools, we analyzed existing systems for narrative writing and structural planning. While many tools offer strong support for either drafting or organization, few successfully bridge the gap between spatial structure and generative AI. Most current solutions force users to choose between a flexible planning environment and AI-assisted text generation, rather than benefiting from both simultaneously.

\textbf{\textit{Sudowrite.}} 
Sudowrite is a prominent AI co-creativity tool that offers features such as "Story Engine" and "Rewrite". While it has recently introduced a canvas for brainstorming, its primary workflow remains largely document-centric and linear. This design encourages users to scroll forward through a single text stream rather than exploring divergent possibilities simultaneously. In practice, it functions more as a text accelerator than as a tool for managing complex narrative structures.

\textbf{\textit{Gingko Writer.}} 
Gingko Writer takes the opposite approach by using a tree-based structure, allowing writers to fracture linear narratives into manageable, hierarchical blocks. While this structure effectively supports non-linear drafting and organization, the tool lacks deep integration with generative AI. Consequently, it serves primarily as a static container for human thought rather than an active, co-creative partner.

\textbf{\textit{Campfire and NovelCrafter.}} Tools like Campfire and NovelCrafter represent a modular approach to creative support. Campfire provides specialized modules for world-building, character tracking, and timelines, integrating AI to accelerate the generation of these specific elements. Similarly, NovelCrafter combines a wiki-style "Codex" with a planning "Matrix". However, these systems typically compartmentalize the creative process: users plan in one view (e.g., a grid or timeline) and write in another (a linear document). IdeaForge distinguishes itself by unifying these activities, allowing the "planning" graph to serve directly as the "drafting" medium.

\subsection{Resources}
% ii. Useful resources
\begin{itemize}
    \item Domain-specific datasets:
    \item Reference implementations of generative systems:
    \item Potentially useful pre-trained models:
\end{itemize}

\subsection{Literature Review}
Recent studies suggest that creative capabilities of LLMs are often overstated. Chakrabarty et al. (2024) found that without significant human guidance, LLMs struggle with rhetorical complexity and tend to default to familiar patterns, creating a "false promise" of creativity. In response, researchers argue that Creative Support Tools (CSTs) should move beyond raw text generation and instead provide guided scaffolding that keeps the human in control of the creative process (Bushnell \& Harrison, 2025). 

This perspective aligns with the framework of Co-Creative Sense-Making (CCSM). Davis and Rafner (2025) describe creativity as a constant shuttling between "clamped" cognition, focused on execution and "unclamped" cognition, concerned with reflection on structure. Standard chat interfaces often disrupt this cycle by forcing users down a linear path. IdeaForge seeks to restore this dynamic by adopting Nakakoji’s (2006) metaphor of "skis", which are technologies that enable new types of experiences previously impossible for the user. 

\section{Implementation}
% c. Implementation

\subsection{Project Type}
% i. What type of project are you trying to implement?
\textit{Indicate which type of project you implemented:}
\begin{itemize}
    \item Proof-of-Concept
    \item (Re)Implementation of Existing Example
    \item Autonomous Installation/Model
\end{itemize}

\subsection{Resources Used}
% ii. What resources did you use in your implementation?
\textit{List the pre-existing resources used. Acknowledge the sources and provide a description of how you have modified and extended them.}

\subsection{Technical Construction}
% iii. What did you implement to build your system?
\textit{Describe the technical details of what you implemented to build your system.}

\section{Example Outputs}
% d. Example Outputs

% i. Provide some examples of the output of your system.
\textit{Provide examples of the output. Discuss how these outputs reflect an exploration of your system's capabilities and/or limitations.}

% Example of how to insert an image:
% \begin{figure}[h]
%     \centering
%     \includegraphics[width=0.8\textwidth]{example_output.png}
%     \caption{An example output generated by the system.}
%     \label{fig:example1}
% \end{figure}

\section{Reflection and Future Work}
% e. Reflection and Future Work

\subsection{Reflection}
% i. Conclude your report with a reflection on what worked and what didn’t.
\textit{Reflect on what worked and what didn’t. Consider how your implementation satisfies the aims set out at the beginning.}

\begin{itemize}
    \item \textbf{Achievements:} What are the significant achievements of your implementation as a creative support tool? (e.g., comparison to original system if applicable).
    \item \textbf{Limitations:} What are the significant limitations of your implementation?
    \item \textbf{Retrospective:} What would you do differently if you were to start your project again now?
\end{itemize}

\subsection{Future Directions}
% ii. Identify some future directions for your project.
\textit{Identify future directions for the project:}

\begin{itemize}
    \item \textbf{Extensions:} How might your creative support tool be extended? (e.g., new domains or datasets).
    \item \textbf{Improvements:} How might your creative support tool be improved?
\end{itemize}

\end{document}